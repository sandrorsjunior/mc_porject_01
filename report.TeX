\documentclass[12pt, a4paper]{article}

% --- PREAMBLE ---
\usepackage[a4paper, top=3cm, bottom=2cm, left=3cm, right=2cm]{geometry}
\usepackage{fontspec}
\usepackage{graphicx} % Required for images
\usepackage{float}    % Required for precise image placement
\usepackage{amsmath}  % Required for math formulas

% Language settings
\usepackage[portuguese, bidi=basic, provide=*]{babel}
\babelprovide[import, onchar=ids fonts]{portuguese}
\babelprovide[import, onchar=ids fonts]{english}

% Fonts
\babelfont{rm}{Noto Sans}
\babelfont{sf}{Noto Sans}

% Code Formatting
\usepackage{listings}
\usepackage{xcolor}
\usepackage{hyperref} % For URLs

% Colors for code
\definecolor{codegreen}{rgb}{0,0.6,0}
\definecolor{codegray}{rgb}{0.5,0.5,0.5}
\definecolor{codepurple}{rgb}{0.58,0,0.82}
\definecolor{backcolour}{rgb}{0.95,0.95,0.95}

% Listing style
\lstdefinestyle{mystyle}{
    backgroundcolor=\color{backcolour},   
    commentstyle=\color{codegreen},
    keywordstyle=\color{magenta},
    numberstyle=\tiny\color{codegray},
    stringstyle=\color{codepurple},
    basicstyle=\ttfamily\footnotesize,
    breakatwhitespace=false,         
    breaklines=true,                 
    captionpos=b,                    
    keepspaces=true,                 
    numbers=left,                    
    numbersep=5pt,                  
    showspaces=false,                
    showstringspaces=false,
    showtabs=false,                  
    tabsize=2,
    frame=single % Adds a frame around the code
}

\lstset{style=mystyle}

% --- DOCUMENT ---
\begin{document}

% --- COVER PAGE ---
\begin{titlepage}
    \centering
    {\large \textbf{INSTITUTO POLITÉCNICO DO CÁVADO E DO AVE (IPCA)}} \\
    \vspace{0.5cm}
    {\large Escola Superior de Tecnologia} \\
    \vspace{0.5cm}
    {\large Curso: Robótica Colaborativa e Inteligência Industrial} \\
    \vspace{4cm}
    
    {\Large \textbf{RELATÓRIO TÉCNICO}} \\
    \vspace{1cm}
    {\large \textbf{Projeto 01: Monitorização de Porta com ATmega328P}} \\
    \vspace{4cm}
    
    \begin{flushright}
        \textbf{Membros da Equipa:} \\
        Sandro Ribeiro \\
        Mafalda Sofia \\
        Daniel Oliveira
    \end{flushright}
    
    \vspace{1cm}
    
    \begin{flushleft}
        \textbf{Disciplina:} Microcontroladores e Sistemas Digitais \\
        \textbf{Professor:} Eng.º Rui Ribeiro \\
        \textbf{Semestre:} 1º Semestre de 2025/2026
    \end{flushleft}
    
    \vfill
    
    {\large Barcelos, Portugal} \\
    {\large \today}
\end{titlepage}

% --- TOC ---
\tableofcontents
\newpage

% --- CONTENT ---

\section{Introdução}
Este relatório documenta o desenvolvimento de um sistema de monitorização de estado de uma porta utilizando o microcontrolador ATmega328P. O projeto visa aplicar conceitos fundamentais de eletrónica digital, manipulação de registos de GPIO (\textit{General Purpose Input/Output}) e programação estruturada em C para sistemas embebidos.

O sistema lê um sensor de posição (simulado por um interruptor) e atua sobre um LED para sinalizar visualmente se a porta se encontra aberta ou fechada.

\section{Metodologia e Dimensionamento de Componentes}
Antes da implementação do código, foi realizado o dimensionamento dos componentes eletrónicos para garantir a segurança e o funcionamento correto do microcontrolador.

\subsection{Especificações do ATmega328P}
De acordo com a folha de dados (\textit{datasheet}) do ATmega328P:
\begin{itemize}
    \item Tensão de operação ($V_{CC}$): 5V.
    \item Corrente máxima absoluta por pino de I/O: 40mA.
    \item Corrente recomendada para operação contínua: $\leq 20mA$.
\end{itemize}

\subsection{Cálculo da Resistência do LED ($R_{LED}$)}
Para o indicador visual, optou-se por um LED genérico.
\begin{itemize}
    \item Tensão direta do LED ($V_{f}$): Aprox. 2.0V (Vermelho/Amarelo).
    \item Corrente desejada ($I_{LED}$): 15mA (suficiente para boa luminosidade sem stressar a porta).
\end{itemize}

Aplicando a Lei de Ohm ($V = R \cdot I$), calculamos o valor do resistor limitador de corrente:

\begin{equation}
    R_{LED} = \frac{V_{CC} - V_{f}}{I_{LED}}
\end{equation}

Substituindo os valores:

\begin{equation}
    R_{LED} = \frac{5V - 2V}{0.015A} = \frac{3V}{0.015A} = 200 \Omega
\end{equation}

\textbf{Decisão:} Para garantir uma margem de segurança e utilizar um valor comercial padrão, optou-se por uma resistência de \textbf{330 $\Omega$}. No esquema final simulado, o valor garante que a corrente permaneça abaixo do limite de 20mA do pino do microcontrolador.

\section{Esquema Elétrico e Simulação}
O circuito foi desenhado e validado utilizando o software de simulação \textit{Proteus ISIS}. 

\begin{figure}[H]
    \centering
    \framebox{\parbox{0.8\textwidth}{\centering
        \vspace{3cm}
        \textbf{Esquema Elétrico (Proteus)} \\
        \small\textit{Substituir por: image\_191803.png}
        \vspace{3cm}
    }}
    \caption{Esquema elétrico simulado no Proteus. O botão está conectado ao PB1 e o LED ao PC5.}
    \label{fig:esquema}
\end{figure}

Como observado na Figura \ref{fig:esquema}, o sistema utiliza uma configuração mínima com o ATmega328P. O LED D1 acende quando a lógica do pino PC5 está em nível alto.

\section{Máquina de Estados Finitos (FSM)}
A lógica de controlo foi implementada utilizando uma Máquina de Estados Finitos. Esta abordagem oferece maior organização do código e facilita a manutenção, evitando o uso excessivo de \textit{flags} globais e estruturas condicionais aninhadas complexas.

\subsection{Estados Definidos}
O sistema é composto por três estados principais, definidos no tipo enumerado \texttt{MachineState}:

\begin{itemize}
    \item \textbf{STATE\_CONFIG:} Estado inicial executado apenas uma vez no arranque (Start-up). É responsável por configurar os registos de direção (DDR) e o estado inicial dos pinos.
    \item \textbf{STATE\_CLOSED (Porta Fechada):} Representa o estado seguro. O LED mantém-se desligado. O sistema monitoriza continuamente o pino PB1 à espera de um nível lógico baixo (GND).
    \item \textbf{STATE\_OPEN (Porta Aberta):} Representa o estado de alerta. O LED é ligado. O sistema monitoriza o pino PB1 à espera que este retorne ao nível lógico alto (VCC).
\end{itemize}

\subsection{Diagrama de Transição de Estados}
O fluxo de execução obedece ao diagrama apresentado na Figura \ref{fig:fsm}.

\begin{figure}[H]
    \centering
    \framebox{\parbox{0.8\textwidth}{\centering
        \vspace{4cm}
        \textbf{Diagrama da Máquina de Estados} \\
        \small\textit{Inserir aqui a imagem gerada: fsm\_diagram.png}
        \vspace{4cm}
    }}
    \caption{Diagrama de transição de estados do sistema.}
    \label{fig:fsm}
\end{figure}

\subsection{Transições e Lógica}
A transição entre estados é governada pela leitura do pino PB1:
\begin{enumerate}
    \item \textbf{CONFIG $\rightarrow$ CLOSED:} Transição incondicional após a configuração dos periféricos.
    \item \textbf{CLOSED $\rightarrow$ OPEN:} Ocorre quando \texttt{gpio\_read(\&PINB, PINB1)} retorna \texttt{false}. Isto indica que o sensor fechou o circuito com o GND (porta abriu fisicamente).
    \item \textbf{OPEN $\rightarrow$ CLOSED:} Ocorre quando \texttt{gpio\_read(\&PINB, PINB1)} retorna \texttt{true}. Devido ao resistor de \textit{pull-up}, isto indica que o sensor está aberto (porta fechou fisicamente).
\end{enumerate}

Foi adicionado um pequeno atraso (\texttt{\_delay\_ms(10)}) no final do ciclo \textit{while} para atuar como um filtro simples de ruído (\textit{debounce}), prevenindo leituras erráticas durante a transição mecânica do botão.

\section{Estrutura do Código e Implementação}

\subsection{Organização do Projeto}
A estrutura de ficheiros separa a lógica de aplicação da camada de abstração de hardware (HAL).
\begin{itemize}
    \item \textbf{src/gpio.c} e \textbf{include/gpio.h}: Drivers reutilizáveis.
    \item \textbf{src/main.c}: Máquina de estados da aplicação.
\end{itemize}

\subsection{Implementação da HAL (gpio.c)}
Para evitar a repetição de manipulação de bits (\textit{bit-banging}) manual no código principal, criou-se uma biblioteca auxiliar.

\noindent\begin{minipage}{\linewidth}
\begin{lstlisting}[language=C, caption={Função de escrita em GPIO}]
void gpio_write(volatile uint8_t *port_reg, uint8_t pin, GpioState state) {
    if (state == GPIO_HIGH) {
        *port_reg |= (1 << pin);  // Define bit como 1
    } else {
        *port_reg &= ~(1 << pin); // Define bit como 0
    }
}
\end{lstlisting}
\end{minipage}

\subsection{Código da Máquina de Estados (main.c)}
Abaixo apresenta-se o trecho principal que implementa a FSM descrita na secção anterior.

\noindent\begin{minipage}{\linewidth}
\begin{lstlisting}[language=C, caption={Implementação da FSM no main.c}]
switch (currentState)
{
case STATE_CLOSED:
    gpio_write(&PORTC, PORTC5, GPIO_LOW); // LED Apagado
    // Se botão pressionado (nível lógico 0)
    if (gpio_read(&PINB, PINB1) == false) {
        currentState = STATE_OPEN;
    }
    break;

case STATE_OPEN:
    gpio_write(&PORTC, PORTC5, GPIO_HIGH); // LED Aceso
    // Se botão solto (nível lógico 1 devido ao Pull-Up)
    if (gpio_read(&PINB, PINB1) == true) { 
        currentState = STATE_CLOSED;
    }
    break;
}
\end{lstlisting}
\end{minipage}

\section{Ferramentas de Desenvolvimento}

\subsection{CMake e Compilação}
Optou-se pelo uso do \textbf{CMake} em vez de IDEs proprietárias (como Atmel Studio ou Arduino IDE) para garantir portabilidade e controlo total sobre as \textit{flags} de compilação.
O ficheiro \texttt{CMakeLists.txt} automatiza a chamada da \textit{toolchain} AVR-GCC e gera o ficheiro \texttt{.hex} necessário para o upload.

\subsection{Versionamento de Código (Git)}
Para garantir a integridade do código e facilitar o trabalho em equipa, o projeto foi gerido utilizando o sistema de controlo de versões \textbf{Git}, hospedado no \textbf{GitHub}.

As principais vantagens observadas foram:
\begin{enumerate}
    \item \textbf{Histórico de Alterações:} Permite reverter para versões anteriores caso uma nova funcionalidade introduza erros.
    \item \textbf{Trabalho Paralelo:} Permite que diferentes membros da equipa trabalhem em ficheiros distintos simultaneamente sem conflitos diretos.
    \item \textbf{Backup na Nuvem:} O código está seguro e acessível remotamente.
\end{enumerate}

O repositório oficial do projeto pode ser acedido através do link:\\
\url{https://github.com/sandrorsjunior/mc_porject_01}

\section{Conclusão}
O projeto atingiu todos os objetivos propostos. A escolha do pino PC5 para o LED, fundamentada na análise do datasheet para evitar conflitos com o RESET (PC6), demonstrou a importância do planeamento de hardware. A implementação da camada de abstração (HAL) e o uso de uma Máquina de Estados resultaram num código limpo, modular e determinístico.

\end{document}